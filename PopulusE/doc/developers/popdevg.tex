\documentclass[12pt]{article}
\title{Populus Guide for Developers}
\author{Lars Roe}
\usepackage{html}
\begin{document}

\maketitle
\newpage

\tableofcontents
\newpage

\part{Overview}

\part{Java Source Code}

\section{Models}

\subsection{Files}
By convention, each end model (not meant to be inherited from) should be in the package \verb edu.umn.ecology.populus.model. \textit{ModelName}.

\subsubsection{Model}
A \verb Model  holds together the basic parts of a model.

\subsubsection{ModelPacket}
A \verb ModelPacket  is a simple wrapper for a model so we can refer to one class at a time, and used in making the menus.  The menus are created in \verb initializeMenuPackets() , and this is manually updated to add or remove models.


\subsubsection{ModelPanel}
The \verb ModelPanel  (input window) base files are in \verb edu.umn.ecology.populus.edwin  (short for editor window, from the Pascal DOS program's naming conventions).

\verb registerChildren()  looks at all of the components, and sets event listeners where appropriate.  Read Events for more information.

\subsubsection{ModelOutputPanel}
The \verb OutputPanel  (output window) base files are in edu.umn.ecology.populus.resultwindow

\subsection{Events}

When changes in the input panel occur, events - or messages - are sent to the output.
The \verb ModelPanel  will call \verb fireModelPanelEvent()  whenever a change occurs, with a constant such as \verb CHANGE_PLOT .  If this warrants a new output, \verb ModelPanel  will be queried for, in the case of Basic Plot, new plot info.
Do not assume that \verb getPlotInfo()  will be called whenever you call \verb fireModelPanelEvent .  For example, if changing the value of a radio button should disable another parameter, that should be done separately from \verb getPlotInfo() .  See the method \verb modelPanelChanged()  to see which events are ignored and which events create a new plot.

Inherited models should not have to worry about when to show the output screen.  \verb registerChildren()  is called after the initialize of the front panel, and this routine looks at all of the components and adds listeners to the ones that should through events.  There is a setting in the User Settings so that users can change when to automatically update the output and making decisions on a model-by-model basis will not work with this.

\subsection{Adding a Model to the Menus}
To add a model to the menu, add a ModelPacket in Preferences.

Lars - it appears this is done in two different spots, hard-coded.  I wanted the ability to be able to dynamically determine files.
What is SelectModelDialog??

\subsection{Basic Model}
Most models will derive from \verb BasicPlotModel , in the plot directory.

\subsection{Common Variants}
Most models extend from \verb edu.umn.ecology.populus.plot.BasicPlotModel , which does basic graphing.  But you don't have to do this.  See \verb Woozleology  for an example of one that does not extend from this.

\section{Main}
\verb main  is found in \verb edu.umn.ecology.populus.core.PopRun .  The \verb DesktopWindow  is the primary GUI background to the application.

\section{Help}
When we click the Help button on a model or the main DesktopWindow, we 
LARS - TODO
The help system was changed dramatically in 5.5, by modifying the local help file to use the language specified by the user's configuration.

\section{Preferences}


\subsection{PreferencesFile}

The file for keeping state is stored as \verb userpref.po  in the user's home directory (as of Populus 5.4).  It is loaded during initialization.  By default, it is in the user's home directory -- not in Populus's -- because we aren't guaranteed write permission for all systems.  This can be overrided by the startup command - see README.config.

\section{GUI Widgets}

\subsubsection{ParameterField}
The ParameterField was originally concocted as a spinner.  But then we added the variable name, and variable information to the parameter.

\section{Javadoc}

I wish the code were better documented.  But you can still use \verb javadoc  to generate documentation for the files.

\part{Installer}
Populus Splash Screen. We have a file called Populus*.*.psd which is a photoshop file describing the title screen. For a new release, we probably want a change in version number, so make a new .psd file with the new version, and then export it to gif format (calling it \verb PopulusSplashScreen.gif ) and replace the one in edu/.../core/ with the new gif.

\part{Web Page}

This should all be handled by the UMN Web team. They now use Drupal (a content management system).  For 5.5, I just gave them a new JAR file.

\part{Test and Verification}
\section{Release Checklist}

Check that help works on all different platforms.


current issues for troubleshooting help file:
on mac os x:
the populus parameter field arrows are dim
screen resolution can cause windows to be smaller than they should be - just resize
on pc:

\section{Platform}
It's a good idea to test on different platforms.
\subsection{Linux}
LiveCD SLAX can boot up Linux on an otherwise Windows computer.  There are other options now too.
\subsection{MacOS}
You really just need a Mac for this.  The UofM computer guys have testers to help with this.


\part{Administration}
\section{Setting Up New Machine}
\subsection{Development Software}
\subsubsection{Eclipse}
Populus now uses Eclipse to develop Populus.

\subsubsection{JClass}
JClass includes the chart software for Java that we use.  The Manifest file in the JAR file they included has some bogus \verb dependson  lines that give warnings when you try to run.  I manually deleted these, and just keep this new version around.
JClass keeps switching companies.  We have an old version of their product, and I don't have any reason for upgrading.
\subsection{Other Software}

\subsubsection{Git, and TortoiseGit}
See the Backing Up section.

\subsubsection{LaTeX}
I use \verb TeXworks to edit LaTeX files.  We don't use LaTeX for any externally-facing file.

\subsubsection{Photoshop}
Use Photoshop to make the pictures for, say, the Web page.  There are saved \verb .psd  files around that contain the source image to work from with its Layers.

\section{Backing Up}
Source uses Git.

\end{document}

